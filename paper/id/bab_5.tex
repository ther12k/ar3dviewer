%==========================================================================================================
% MULAI BAB V
%==========================================================================================================
\chapter{KESIMPULAN DAN SARAN}
\label{chap:penutup}
%==========================================================================================================
% Subbab
%==========================================================================================================
\section{Kesimpulan}
\label{sec:kesimpulan}
Berdasarkan pembahasan bab-bab sebelumnya dan didukung oleh hasil pengujian, dapat diambil kesimpulan sebagai berikut.
\begin{enumerate}
\item Aplikasi dapat berjalan dengan baik tanpa perlu menginstall aplikasinya, karena aplikasi diakses menggunakan browser yang mempunyai \textit{plug-in} flash. 
\item Aplikasi dapat merender objek 3D dengan format DAE dan 3DS sesuai dengan marker yang didteksi, akan tetapi tingkat pengenalan marker sangat dipengaruhi cahaya dan bentuk marker yang digunakan.
\item Aplikasi dapat merender objek 3D sederhana dengan baik dengan nilai FPS yang relatif baik (24 sampai 50). 
\item Untuk objek 3D kompleks aplikasi tidak dapat berjalan dengan baik, terjadi \textit{flicker} atau gambar tidak bergerak sama sekali.
\end{enumerate}

\section{Saran}
\label{sec:saran}
\begin{enumerate}
\item Untuk pengembangan selanjutnya, diharapkan dapat menggunakan \textit{library} yang lebih baik. Marker yang digunakan diharapkan lebih variatif, dengan pola yang lebih menarik atau mungkin dengan pola yang berwarna sehingga lebih menarik.
\item Perangkat keras yang digunakan tidak lagi komputer tapi berupa \textit{gadget} yang lebih kecil dan bisa menjalankan flash, sehingga aplikasi benar-benar dapat dijalankan dari mana saja. 
\end{enumerate}